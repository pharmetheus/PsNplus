\makeatletter
\def\input@path{{/PMX/Projects/Pharmetheus/PMX-RRW-PMX-2/ProjectManagement/LaTeX/}}
\makeatother
\documentclass[hideglossary,notoc,hidelof,hidelot,hideTheSignaturePage,hideLinkCurrent,hideloa,pdfLatex,noClient,notitle,hideConfidential]{PMXstyle-20190820}

%%% code for bug in section numbering titlesec version 2.10.1
%%% https://tex.stackexchange.com/questions/299969/titlesec-loss-of-section-numbering-with-the-new-update-2016-03-15
\usepackage{etoolbox}
\makeatletter
\patchcmd{\ttlh@hang}{\parindent\z@}{\parindent\z@\leavevmode}{}{}
\patchcmd{\ttlh@hang}{\noindent}{}{}{}
\makeatother

%%% Project information
%\projectsubtitle{}
\reference{PMX-SCM-PMX-1}
\documentversion{1.0.7}
\documentdate{\today}
\confidential{}
\documenttype{User documentation}
\projecttitle{scmreport}
\shortprojecttitle{scmreport} 
\documentreference{scmreport}
\pmtcontact{Kajsa Harling}
\phonepmtcontact{+46 73 593 2055}
\emailpmtcontact{kajsa.harling@pharmetheus.com}
%% Pharmetheus document signer
%\documentauthorA{Kajsa Harling}


\newacronym{ASR}{ASR}{Adaptive Scope Reduction scm}
\newcommand{\guidetoolname}{scmreport}

\newenvironment{optionlist}{
\renewcommand{\arraystretch}{1.1}
\setlength{\leftmargini}{2.5cm}
\begin{description}
%\setlength{\itemsep}{0ex}
}
{\end{description}}

\newcommand{\optname}[1]{\item{{\bfseries\texttt-#1}\newline}}
\newcommand{\optdefault}[2]{\item{{\bfseries\texttt-#1}{\mbox{ = \it #2}}\newline}}
% optconfig is for options that can only be set in a (scm) config file, always without -
\newcommand{\optconfig}[2]{\item{{\bfseries\texttt#1}{\mbox{ = \it #2}}\newline}}

\newcommand{\nextopt}{}

\begin{document}
\section{Summary}
The scmreport script is used for summarizing the results of an scm or scmplus run.
The examples below show output for an scmplus run.
The output of scmreport is a summary of the contents of the scm log file. 
NONMEM output files are not parsed. 

\section{Output with default settings}
\begin{verbatim}
scmreport run11.scm
\end{verbatim}
{\tiny
\begin{verbatim}
Model            N_test    N_ok N_localmin N_failed StepSelected StepStashed StepReadded BackstepRemoved
CLAPGR-2              4       2          2        0       -           1           3               -
CLCV1-2               2       2          0        0       3           1           3               2
CLWGT-2               2       2          0        0       2           -           -               3
VCV2-2                3       3          0        0       4           1           3               1
VCVD1-2               4       3          1        0       -           1           3               -
VWGT-2                1       1          0        0       1           -           -               -
\end{verbatim}
}


{\bfseries Model}\\
The name of the model in the scm log file, which is composed of the parameter, covariate and state number.

{\bfseries N\_test}\\
The number of times, i.e. in how many iterations, 
the parameter-covariate relation has been tested.
%fixme backward counted?

{\bfseries N\_ok}\\
How many times, out of N\_test, does the scm log file list an ofv that such that extended model
has lower ofv than the base model?
%fixme

{\bfseries N\_localmin}\\
How many times, out of N\_test, did the run end in a local minimum, i.e. the log file lists a
higher ofv for the extended model than for the base model?

{\bfseries N\_failed}\\
How many times, out of N\_test, did the scm log file state that the model failed?

{\bfseries StepSelected} \\
In which forward step, if any, was the parameter-covariate relation selected?

{\bfseries StepStashed} \\
Only for scmplus: In which forward step, if any, was the parameter-covariate relation stashed, i.e. removed from the set of relations tested again?

{\bfseries StepReadded} \\
Only for scmplus: In which forward step, if any, was a previously stashed
parameter-covariate relation retested?

{\bfseries BackstepRemoved}\\ 
Both for scm and scmplus: 
In which backward step, if any, was a previously selected parameter-covariate relation removed?


\newpage

\section{Output with option -ofv\_table}
\begin{verbatim}
scmreport run11.scm -ofv
\end{verbatim}
{\tiny
\begin{verbatim}
Step             covariate  parameter  state         BASEOFV          NEWOFV         OFVDROP   dDF    PVALrun  pvalue
Forward 1              WGT          V      2       725.60200       685.37770        40.22430     1   2.26e-10    0.05
Forward 2              WGT         CL      2       685.37770       663.09532        22.28238     1   0.000002    0.05
Forward 3              CV1         CL      2       663.09532       653.70237         9.39295     1   0.002178    0.05
Forward 4              CV2          V      2       653.70237       647.68823         6.01414     1   0.014192    0.05
Forward 5                                                  -               -               -                -        
Backward 1             CV2          V      1       647.68823       648.16473        -0.47650    -1   0.490010    0.01
Backward 2             CV1         CL      1       648.16473       643.42136         4.74337    -1       9999    0.01
Backward 3             WGT         CL      1       643.42136       648.26975        -4.84839    -1   0.027672    0.01
Backward 4                                                 -               -               -                -        
Final included         WGT          V      2       648.26975       749.50873      -101.23898    -1   8.15e-24        
\end{verbatim}
}



{\bfseries Step}\\
The search direction and the step number in that direction. The last rows list the included relations
at the end of the scm or scmplus run.

{\bfseries covariate}\\
The name of the covariate.

{\bfseries parameter}\\
The name of the parameter.

{\bfseries state}\\
The scm state number.

{\bfseries BASEOFV}\\
Base ofv as read from the scm log file for the step.

{\bfseries NEWOFV}\\
New ofv for the selected model as read from the scm log file for the step.

{\bfseries OFVDROP}\\
Drop between BASEOFV and NEWOFV as read from the scm log file.

{\bfseries dDF}\\
The change in degrees of freedom when going from base model to new model.

{\bfseries PVALrun}\\
The actual p-value, computed based on change in ofv, for choosing the extended model.
In a forward step the new model is the extended model and in a backward step the base model is the extended model.

{\bfseries pvalue}\\
The p-value cutoff for selecting/excluding a relation as read from the scm log file.

\newpage

\section{Input and options}

\subsection{Required input}
The name of an scm or scmplus run directory is a required argument.

\subsection{Optional input}

\begin{optionlist}
\optname{ofv\_table}
Show ofv etc for selected relations, instead of the default which is
a summary of all tested relations.
\nextopt
\optdefault{decimals}{N}
Only used in combination with option -ofv\_table.
Number of decimals to display for OFV columns. Default is 5.
\nextopt
\optname{toR}
Print results as data frame code. Default not set.
\nextopt
\optname{logfile}
Name of scm logfile in the directory. 
Default is the name of the logfile in
version\_and\_option\_info.txt, 
so normally this option is only needed if version\_and\_option\_info.txt is not present,
or if the entire folder has been moved so that version\_and\_option\_info.txt cannot be used.
\nextopt
\optname{help}
Print help text and exit.
\nextopt
\optname{version}
Print version information and exit.
\nextopt
\end{optionlist}




\end{document}


{\tiny
\begin{verbatim}
MODEL            TEST     BASE OFV     NEW OFV         TEST OFV (DROP)    GOAL     dDF    SIGNIFICANT PVAL
CLAPGR-2         PVAL    742.05105   721.58702             20.46402  >  18.30700   10        YES!  0.025157 
CLWGT-2          PVAL    742.05105    FAILED                 FAILED  >   3.84150    1                    999
VAPGR-2          PVAL    742.05105   726.38092             15.67013  >  18.30700   10              0.109470 
VWGT-2           PVAL    742.05105    FAILED                 FAILED  >   3.84150    1                    999

Parameter-covariate relation chosen in this forward step: CL-APGR-2
CRITERION              PVAL < 0.05
BASE_MODEL_OFV         742.05105
CHOSEN_MODEL_OFV       721.58702
Relations included after this step:
CL      APGR-2  
V       
--------------------
\end{verbatim}
}

